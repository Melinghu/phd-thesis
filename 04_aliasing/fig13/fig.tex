% Algorithm to min. and max. tangential component of the local wavenumber vector 

%*****************************************************************************
% Copyright (c) 2019      Fiete Winter                                       *
%                         Institut fuer Nachrichtentechnik                   *
%                         Universitaet Rostock                               *
%                         Richard-Wagner-Strasse 31, 18119 Rostock, Germany  *
%                                                                            *
% This file is part of the supplementary material for Fiete Winter's         *
% PhD thesis                                                                 *
%                                                                            *
% You can redistribute the material and/or modify it  under the terms of the *
% GNU  General  Public  License as published by the Free Software Foundation *
% , either version 3 of the License,  or (at your option) any later version. *
%                                                                            *
% This Material is distributed in the hope that it will be useful, but       *
% WITHOUT ANY WARRANTY; without even the implied warranty of MERCHANTABILITY *
% or FITNESS FOR A PARTICULAR PURPOSE.                                       *
% See the GNU General Public License for more details.                       *
%                                                                            *
% You should  have received a copy of the GNU General Public License along   *
% with this program. If not, see <http://www.gnu.org/licenses/>.             *
%                                                                            *
% http://github.com/fietew/phd-thesis                 fiete.winter@gmail.com *
%*****************************************************************************

\begin{algorithmic}[1]
  \Function{MinMaxWavenumberCircle}{$\sfcircleh, \sfpossec$}
  \State $\sfposh, \sfRh \gets \sfcircleh$
  \Comment centre and radius of circle
  \State $\varrho_{\mathrm h} \gets 
  \nicefrac{\sfRh}{|\sfposh-\sfpossec|}$
  \Comment \eqref{eq:derivations_circle_ratio}
  \State $\sfknorm_{\mathrm h, \sftangentsec} \gets
  \sfknorm_{\sfgreens,\sftangentsec}(\sfposh - 
  \sfpossec)$
  \Comment \eqref{eq:derivations_circle_center_tangential}
  \If {$
    \varrho_{\mathrm h} > 1 \,\textbf{or}
    - \sqrt{1-\varrho_{\mathrm h}^2} > \hat \sfk_{\mathrm h, \sftangentsec}
    $}
  \Comment \eqref{eq:derivations_circle_kminmax_allcases}
  \State $\sfknorm^{\min}_{\sfgreens, \sftangentsec} \gets -1$
  \Else
  \State $\sfknorm^{\min}_{\sfgreens, \sftangentsec} \gets 
  \sfknorm_{\mathrm h, \sftangentsec} \sqrt{1-\varrho_{\mathrm h}^2}
  - \varrho_{\mathrm h} \sqrt{1 - \sfknorm_{\mathrm h, \sftangentsec}^2}  
  $
  \Comment \eqref{eq:derivations_circle_kminmax_case1}
  \EndIf
  \If {$
    \varrho_{\mathrm h} > 1 \,\textbf{or}
    + \sqrt{1-\varrho_{\mathrm h}^2} > \hat \sfk_{\mathrm h, \sftangentsec}
    $}
  \Comment \eqref{eq:derivations_circle_kminmax_allcases}
  \State $\sfknorm^{\max}_{\sfgreens, \sftangentsec} \gets +1$
  \Else
  \State $\sfknorm^{\max}_{\sfgreens, \sftangentsec} \gets 
  \sfknorm_{\mathrm h, \sftangentsec} \sqrt{1-\varrho_{\mathrm h}^2}
  + \varrho_{\mathrm h} \sqrt{1 - \sfknorm_{\mathrm h, \sftangentsec}^2} $
  \Comment \eqref{eq:derivations_circle_kminmax_case1}
  \EndIf
  \State \Return $\sfknorm^{\min}_{\sfgreens, \sftangentsec},
  \sfknorm^{\max}_{\sfgreens, \sftangentsec}$
  \EndFunction
\end{algorithmic}
