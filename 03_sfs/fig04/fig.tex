% sketch for 2.5D synthesis

%*****************************************************************************
% Copyright (c) 2019      Fiete Winter                                       *
%                         Institut fuer Nachrichtentechnik                   *
%                         Universitaet Rostock                               *
%                         Richard-Wagner-Strasse 31, 18119 Rostock, Germany  *
%                                                                            *
% This file is part of the supplementary material for Fiete Winter's         *
% PhD thesis                                                                 *
%                                                                            *
% You can redistribute the material and/or modify it  under the terms of the *
% GNU  General  Public  License as published by the Free Software Foundation *
% , either version 3 of the License,  or (at your option) any later version. *
%                                                                            *
% This Material is distributed in the hope that it will be useful, but       *
% WITHOUT ANY WARRANTY; without even the implied warranty of MERCHANTABILITY *
% or FITNESS FOR A PARTICULAR PURPOSE.                                       *
% See the GNU General Public License for more details.                       *
%                                                                            *
% You should  have received a copy of the GNU General Public License along   *
% with this program. If not, see <http://www.gnu.org/licenses/>.             *
%                                                                            *
% http://github.com/fietew/phd-thesis                 fiete.winter@gmail.com *
%*****************************************************************************

\documentclass{article}

%% FONTS
\usepackage[osf,sc]{mathpazo}

\makeatletter
\renewcommand\normalsize{\@setfontsize\normalsize\@xpt{14}\abovedisplayskip 
  10\p@ \@plus2\p@ \@minus5\p@ \abovedisplayshortskip \z@ \@plus3\p@ 
  \belowdisplayshortskip 6\p@ \@plus3\p@ \@minus3\p@ \belowdisplayskip 
  \abovedisplayskip \let\@listi\@listI}
\normalbaselineskip=14pt
\normalsize
\renewcommand\small{\@setfontsize\small\@ixpt{12}\abovedisplayskip 8.5\p@ 
  \@plus3\p@ \@minus4\p@ \abovedisplayshortskip \z@ \@plus2\p@ 
  \belowdisplayshortskip 4\p@ \@plus2\p@ \@minus2\p@ 
  \def\@listi{\leftmargin\leftmargini \topsep 4\p@ \@plus2\p@ \@minus2\p@ 
    \parsep 
    2\p@ \@plus\p@ \@minus\p@ \itemsep \parsep}\belowdisplayskip 
  \abovedisplayskip}
\renewcommand\footnotesize{\@setfontsize\footnotesize\@viiipt{10}\abovedisplayskip
  6\p@ \@plus2\p@ \@minus4\p@ \abovedisplayshortskip \z@ \@plus\p@ 
  \belowdisplayshortskip 3\p@ \@plus\p@ \@minus2\p@ 
  \def\@listi{\leftmargin\leftmargini \topsep 3\p@ \@plus\p@ \@minus\p@ \parsep 
    2\p@ \@plus\p@ \@minus\p@ \itemsep \parsep}\belowdisplayskip 
  \abovedisplayskip}
\renewcommand\scriptsize{\@setfontsize\scriptsize\@viipt\@viiipt}
\renewcommand\tiny{\@setfontsize\tiny\@vpt\@vipt}
\renewcommand\large{\@setfontsize\large\@xipt{15}}
\renewcommand\Large{\@setfontsize\Large\@xiipt{16}}
\renewcommand\LARGE{\@setfontsize\LARGE\@xivpt{18}}
\renewcommand\huge{\@setfontsize\huge\@xxpt{30}}
\renewcommand\Huge{\@setfontsize\Huge{24}{36}}
\makeatother

%%
\usepackage{soundfield}
\newcommand{\ft}[0]{\footnotesize}
\newcommand{\scs}[0]{\scriptsize}
\newcommand{\sm}[0]{\small}
\sfrenewsymbol{wc}{\frac{\sfomega}{\mathrm{c}}}

%% COLORS
\definecolor{activecolor}{RGB}{150, 150, 150}
\definecolor{area}{RGB}{236, 236, 236}
\definecolor{local}{RGB}{254, 204, 0}
\definecolor{rostock-uni}{RGB}{0,74,153}

%% LENGTHS
\newlength{\dissfullwidth}
\newlength{\disstextwidth}
\newlength{\dissmarginwidth}
\setlength{\dissfullwidth}{16.46 cm}
\setlength{\disstextwidth}{10.7 cm}
\setlength{\dissmarginwidth}{4.94 cm}

%% TikZ
\usepackage{tikz}%  TikZ for drawing sketches
\usetikzlibrary{%  TikZ libraries
  decorations.pathreplacing,%
  decorations.markings,%
  calc,%
  arrows,%
  through,%
  intersections,%
  positioning,%
  external}
\usepackage{audioicons}%  package for loudspeakers, microphones, etc.

%%% TikZ styles
\tikzstyle{loudspeaker} = [%  style for loudspeaker
  basic loudspeaker, 
  draw=black!70, 
  fill=white, 
  minimum height=3pt,
  minimum width=1.5pt,
  inner sep=0.5pt,
  relative cone width=1.5,
  relative cone height=2.5
]

\tikzstyle{focused} = [%  style for focused source
  circle,
  fill=activecolor,
  draw=black,
  thin,
  inner sep=0,
  minimum width=0.15cm
]

%%% TikZ commands
% add labeled coordinate to position on contour
% usage: \draw[mark coordinate (<label>) at <position>] ...
% inputs:
%   label     - label for coordinate
%   position  - relative position (0.0 for start, 1.0 for end)
\tikzset{
  mark coordinate/.style args={(#1) at #2}{
    postaction={
      decorate,
      decoration={
        markings,
        mark=at position #2 with {\coordinate (#1);}
      }
    }
  }
}
% add node to position on contour
% usage: \draw[mark node (<label>) at <position> with {<args>}] ...
% inputs:
%   label     - label for node
%   position  - relative position 0.0 for start, 1.0 for end)
%   args      - name value pairs compatible with \node[ args ]
\tikzset{
  mark node/.style args={(#1) at #2 with #3}{
    postaction={
      decorate,
      decoration={
        markings,
        mark=at position #2 with {\node[#3](#1) {};}
      }
    }
  }
}
% place loudspeaker symbols uniformly along on contour
% usage: \draw[add loudspeakers <number>] ...
% inputs:
%   number    - number of loudspeakers
\tikzset{
  add loudspeakers/.style args={#1}{
    postaction={
      decorate,
      decoration={
        markings,
        mark=between positions 0 and 1 step 1/#1 with {%
          \node[loudspeaker,
          fill=activecolor,
          transform shape,
          rotate=90,
          anchor=cone] {};
        },
      }
    }
  }
}

% place focused source symbols uniformly along on contour
% usage: \draw[add focused <number>] ...
% inputs:
%   number    - number of focused source
\tikzset{
  add focused/.style args={#1}{
    postaction={
      decorate,
      decoration={
        markings,
        mark=between positions 0 and 1 step 1/#1 with {%
          \node[focused] {};
        },
      }
    }
  }
}

% draw right angle
\def\dotMarkRightAngle[size=#1](#2,#3,#4){%
  \draw ($(#3)!#1!(#2)$) --
  ($($(#3)!#1!(#2)$)!#1!90:(#2)$) --
  ($(#3)!#1!(#4)$);
  \path (#3) --node[circle,fill,inner sep=.5pt]{}
  ($($(#3)!#1!(#2)$)!#1!90:(#2)$);
}
%  for TikZ styles and functions

\usepackage{tikz-3dplot} %for tikz-3dplot functionality

% generates a tightly fitting border around the work
\usepackage[active,tightpage]{preview}  
\PreviewEnvironment{tikzpicture}
\setlength\PreviewBorder{0mm}

\sfrenewsymbol{cylphi}{\alpha}
\sfrenewsymbol{sphphi}{\alpha}
\sfrenewsymbol{sphtheta}{\beta}

\usetikzlibrary{fadings}

\begin{document}
\tdplotsetmaincoords{70}{70}

\pgfdeclarelayer{bottom}
\pgfdeclarelayer{top}
\pgfsetlayers{bottom,main,top} 
%

\pgfmathsetmacro{\projleft}{0.055}
\pgfmathsetmacro{\projright}{0.51}
\pgfmathsetmacro{\xo}{0.65}
\pgfmathsetmacro{\plotscale}{4.5}


\tikzstyle{fade} = [draw=none, fill=none,left color=black!80, right 
color=black!20]
\tikzstyle{plane} = [draw=none, fill=black!10]
\tikzstyle{plane fade} = [draw = none, fill=none, left color=black!70, right 
color=black!10]

\begin{tikzpicture}[tdplot_main_coords,inner sep = 2pt]

% coordinates origin
\coordinate (origin) at (-0.25,-0.5,0);

\draw[fill=black!30] (origin) -- +(5,0,0) -- +(5,3.25,0) -- +(0,3.25,0) -- 
cycle;

% coordinate axes
\draw[thick, -latex] (origin) -- +(1,0,0) node[anchor=north east]{$\sfx$};
\draw[thick, -latex] (origin) -- +(0,1,0) node[anchor=south]{$\sfy$};
\draw[thick, -latex] (origin) -- +(0,0,1) node[anchor=south]{$\sfz$};

\draw[fill=white,
draw=black,%
thick,%
mark coordinate=(proj_left) at \projleft,
mark coordinate=(proj_right) at \projright,
mark coordinate=(x0) at \xo,
mark coordinate=(boundary) at 0.25,
]
plot[scale=\plotscale] file {fig.csv} -- cycle;
%

% labels
\node[below left=2pt of boundary, inner sep=0] (boundary_label) {$\sfcontour$};
% \draw[-latex'] (boundary_label) to[out=-45,in=-135] (boundary);

% bottom layer
\coordinate (Shift) at (0,0,-1.75);
\coordinate (proj_left_bottom) at ($(proj_left)+(Shift)$);
\coordinate (proj_right_bottom) at ($(proj_right)+(Shift)$);

\begin{pgfonlayer}{bottom}
  \tdplotsetrotatedcoordsorigin{(Shift)}
  
  \draw[fade]
    (proj_right_bottom) --
    (proj_left_bottom) --
    (proj_left) --
    (proj_right) -- 
    cycle;

  \begin{scope}
    \clip (proj_right_bottom) --
      (proj_left_bottom) --
      (proj_left) --
      (proj_right) -- 
      cycle;
    
    \draw[densely dotted, tdplot_rotated_coords]
      plot[scale=\plotscale] file {fig.csv} -- cycle;
  \end{scope}
  
  \begin{scope}
    \begin{pgfinterruptboundingbox}
      \clip (proj_right_bottom) --
      (proj_left_bottom) --
      ($(proj_left_bottom)+(0,0,-10)$) -- 
      ($(proj_right_bottom)+(0,0,-10)$) -- 
      cycle;
    \end{pgfinterruptboundingbox}
    
    \draw[fade, tdplot_rotated_coords]
      plot[scale=\plotscale] file {fig.csv} -- cycle;
    \draw[tdplot_rotated_coords]
      plot[scale=\plotscale] file {fig.csv} -- cycle;
  \end{scope}
    
  \draw (proj_left) -- (proj_left_bottom);
  \draw (proj_right) -- (proj_right_bottom) node[midway,right]{$\sfboundary$};
  \draw[densely dotted] (x0) -- +(Shift); 
  %\clip (proj_left) -- plot[scale=\plotscale] file {fig.csv} -- cycle;
  %\draw plot[scale=\plotscale] file {fig.csv} -- cycle;
  
  \draw[plane fade, tdplot_rotated_coords]
    plot[scale=\plotscale] file {fig.csv} -- cycle;
  
  \node at ($(2.5,1,0)+(Shift)$) {$\sfset$};
\end{pgfonlayer}

% top layer
\coordinate (Shift) at (0,0,1.75);
\coordinate (proj_left_top) at ($(proj_left)+(Shift)$);
\coordinate (proj_right_top) at ($(proj_right)+(Shift)$);

\begin{pgfonlayer}{top}
  \draw[fade]
    (proj_right_top) --
    (proj_left_top) --
    (proj_left) --
    (proj_right) -- 
    cycle;

  \begin{scope}
    \begin{pgfinterruptboundingbox}
      \clip (proj_right) --
      (proj_left) --
      ($(proj_left)+(0,0,+10)$) -- 
      ($(proj_right)+(0,0,+10)$) -- 
      cycle;
    \end{pgfinterruptboundingbox}
    \draw[draw=black, densely dotted]
      plot[scale=\plotscale] file {fig.csv} -- cycle;
  \end{scope}

  \tdplotsetrotatedcoordsorigin{(Shift)}
  \draw[plane, draw=black, tdplot_rotated_coords]
    plot[scale=\plotscale] file {fig.csv} -- cycle;
  
  \draw (proj_left) -- (proj_left_top);
  \draw (proj_right) -- (proj_right_top);
  
  \draw[plane fade]
    plot[scale=\plotscale] file {fig.csv} -- cycle;
    
  \draw[densely dotted] (x0) -- +(Shift); 
  
  \begin{scope}
    \clip plot[scale=\plotscale] file {fig.csv} -- cycle;
    \draw[densely dotted] (x0) -- +($(0,0,0)-(Shift)$);
  \end{scope}
    
  \node[fill=black, circle, inner sep=1pt] (x) at (1,1,0) {};
  \node[right=2pt of x, inner sep=0] {$\sfpos$};
  
  \node[fill=black, circle, inner sep=1pt] (x0_point) at (x0) {};
  \node[below left=2pt of x0_point, inner sep=0] {$\sfpossec$};
  
  \node at (2.5,1,0)  {$\sfregion$};
\end{pgfonlayer}

\end{tikzpicture}
\end{document}
